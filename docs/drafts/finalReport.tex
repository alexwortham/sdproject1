\documentclass[12pt,letterpaper]{article}
\usepackage[ampersand]{easylist}
\usepackage[margin=1in]{geometry}
\usepackage{requirements}
\usepackage{float}
\bibliography{Bibliography.bib}
%% @DocumentRequirement (1.1,"Title")
\title{Final Report}
%% @DocumentRequirement (1.3,"Team #")
\team{1}
%% @DocumentRequirement (1.4,"Team Members")
\author{Alex Chaloux, Alex Wortham, Asanga Bandara, Doug Bouler}
%% @DocumentRequirement (1.5,"Customer")
\customer{Vasile Tudor Garbulet}

\begin{document}

%% @DocumentRequirement (1,"Title Page")
%% @DocumentRequirement (1.2,"Date")
\reqstitlepage

%% @DocumentRequirement (2,"Table of Contents")
\tableofcontents
\clearpage
\setstretch{2}

%% @DocumentRequirement (3,"Executive Summary")
\section{Executive Summary}

%% @DocumentRequirement (3.1,"Concise statement of the need addressed and the project's goals.")
%% @DocumentRequirement (3.2,"Summary of the results and whether the goals were met.")
%% @DocumentRequirement (3.3,"Brief Discussion of challenges and outcomes.")

3.1 Concise statement of the need addressed and the project's goals.

3.2 Summary of the results and whether the goals were met.

3.3 Brief Discussion of challenges and outcomes.

%% @DocumentRequirement (4,"Requirements")
\section{Requirements}

%% @DocumentRequirement (4.1,"Numbered list of requirements initially agreed to by team and customer.")
\begin{easylist}[articletoc] \requirements

& Operation Requirements

	&& The mower must be battery powered.

		&&& The battery must be rechargeable.

	&& The mower must be autonomous as outlined in requirement \ref{autonomous}.
	&& A home base will be provided for the mower.


& \label{user interaction}User Interaction Requirements

	&& The user may issue the following commands to the mower.

		&&& Begin mowing operations, as specified in requirement \ref{begin}.
		&&& Stop immediately, as specified in requirement \ref{stop}.
		&&& Return to base, as specified in requirement \ref{rtb}.

	&& The following state information will be provided to the user.

		&&& Current battery level.
		&&& Battery is charging.
		&&& Battery has completed charging.
		&&& Mowing operation is in progress.
		&&& Desired grass height.
		&&& Days of week the mower is scheduled to run.
		&&& Time of each day the mower is scheduled to run.
		&&& Mower operation has been forcefully stopped because of a violation in requirements outlined in \ref{environment} and \ref{safety}.

	
& \label{autonomous}Autonomous Functions Requirements

	&& The mower must return to base when it detects low battery level.
	&& \label{rtb}Return to base on command from interface.
	&& \label{stop}Stop immediately on command from interface. The following conditions define the stopped state.
		&&& Electrical power is disconnected from the wheels.
		&&& Electrical power is disconnected from the blades.
	&& \label{begin}Begin mowing on command from interface.
	&& Begin mowing on schedule.

& \label{environment}Operating Environment Requirements

	&& The product must be able to operate normally within an environment that

		&&& is above 32 degrees Fahrenheit.
		&&& is below 115 degrees Fahrenheit.
		&&& \label{incline limits}has ground which the mower detects is less than 33\% incline.
	&& The device will function as specified when it is not in contact with snow.
	&& The device will function as specified when it is not in contact with ice.

& \label{weather}Weather Resistance Requirements.
	
	&& The product's electrical components must be protected by a NEMA Type 3R enclosure. (\href{https://www.nema.org/Products/Documents/nema-enclosure-types.pdf}{https://www.nema.org/Products/Documents/nema-enclosure-types.pdf})
	&& The product's components must comply with International Protection Marketing Code 25 (IP25. \href{http://www.dsmt.com/pdf/resources/iprating.pdf}{http://www.dsmt.com/pdf/resources/iprating.pdf}).

& \label{safety}Safety Requirements

	&& The mower must stop immediately (requirement \ref{stop}) when...

		% These requirements need to be refined.
		&&& it senses an foreign object in its path. 
			&&&& A foreign object is any object small enough that the mower might be capable of running over it, such as garden hoses, small gardening tools, tennis balls, etc.
		&&& an emergency kill switch is pressed.
		&&& removed from its configured boundaries.

	&& The mower will sound an alarm when removed from its configured boundaries.

& Product Configuration Requirements

	&& All of the below must be configurable one or more times.

		&&& \label{boundaries}The user can set operating boundaries for the mower.
		&&& \label{desired height}The user can set desired height of grass.
		&&& The days of week the mower will run.

			&&&& Multiple days may be selected.

		&&& The time of day the mower will run.

			&&&& One time must be selected for each day.

& Budget Requirements

	&& The total cost of the project including, but not limited to, the following shall not exceed \$2,500.  % Here We can say,insted of having project cost, "The total cost of the product would be less than $2,500." ( something like that) 
		&&& Labor.
		&&& Prototypes.
		&&& Final product.
		&&& Unforeseen costs.

\end{easylist}

%% @DocumentRequirement (5,"Change Log")
\section{Change Log}

%% @DocumentRequirement (5.1,"Chronological list of any agreed-upon changes to the requirements.")
5.1 Chronological list of \textit{any} agreed-upon changes to the requirements.

%% @DocumentRequirement (6,"Documentation of the Design Process")
\section{Design Process}

%% @DocumentRequirement (6.1,"Narrative describing how the design proceeded from analysis of the requirements to final result.")

%% @DocumentRequirement (6.1.1,"Describe how the effort was decomposed into manageable pieces to address the agreed-upon requirements.")
With these issues in mind we were able to develop a clear division of labor.  We
divided up our efforts into the categories of user experience, autonomous
design, and mechanical design.  Our lead software developer, Alex Wortham, in
cooperation with solutions architect, Alex Chaloux, took responsibility of
evaluating various autonomous solutions and their implications on user
experience.  Our team lead and hardware designer, Doug Bouler, looked into the
impacts that the mechanical design would have on our electronics platforms as
well as the implementation of these two.  These members' success was facilitated
by our skilled lead researcher and hardware tester, Asanga Bandara, who also
assisted all of us in our tasks.

%% @DocumentRequirement (6.1.2,"What open questions had to be answered by research?")
%% @DocumentRequirement (6.1.3,"What alternatives were explored?")
Of the many questions posed by these challenges, we were most perplexed by how
we would simplify product configuration.  Configuration of our competitor's
products takes in the order of hours, and it was our goal to reduce the
configuration process to the order of minutes.  The key to solving this complex
problem was solving the considerably more complex problem of high precision
localization of a mobile robotics platform.  Research efforts quickly revealed
there is no de facto standard for localization, and furthermore that there is no
simple solution to this highly complex problem.

Research turned up dozens of scholarly and IEEE papers detailing solutions with
varying levels of success but generally poor precision.  Precision localization
of our product is absolutely vital to compete with the precision of the boundary
wire solution.  Finally we found a solution which can be adapted to achieve the
necessary level of precision without adding high costs. \autocite{mobilegps}  We
believe that by supplementing the odometry found in this method with
accelerometers and gyroscopes we can achieve significantly improved drift
compensation. Granted this ability for high precision localization, we arrived
at a simple configuration process that relies on recording the robot's locations
along a path traversed in a user assisted learning mode.

Our next line of questioning was regarding obstacle avoidance.  What are the
best methods of obstacle detection?  Which methods can be implemented within our
cost constraints?  We considered options such as computer vision, infrared,
ultrasonic, and simple mechanical sensors.  Each of these options present their
own challenges, especially in outdoors environments.  Computer vision and
infrared both rely upon certain characteristic properties of light reflecting
off of obstacles in their detection range.  These properties are too easily
impacted by changes in lighting conditions which are inherent to outdoors
environments, making it difficult to implement a solution which can provide
consistent detection.

Our preliminary research suggests that ultrasonic sensors will be able to
provide the most consistent obstacle detection.  Simple mechanical sensors alone
have proven to be insufficient by our competitors. Even so our design will most
likely incorporate some form of mechanical sensors as a fallback detection
system, but more research on the shortcomings of ultrasonic sensors will be
required in order to assess what the best supplemental detection system would
be.

Another area of concern was the selection of a cutting blade which is effective
and safe.  Most existing autonomous products use a fixed blade system similar to
those found on most human operated mowers, and thereby inherit the same safety
concerns.  In our research we determined that this definitely is not the focus
area of innovation for autonomous mowers. The \textit{Husqvarna
Automower}\autocite{automower} however stands out with an innovative blade
design which, if allowed by applicable intellectual property laws, we would
borrow upon for our own design.

Weather proofing is a difficult subject with any product, as nothing can be
truly weather \textit{proof}.  In a search for a more reasonable, testable set
of requirements for weather \textit{resistance}, we sought out publicized
industry standards to suit our product's needs.  Research led us to two widely
accepted standards published by the National Electrical Manufacturers
Association (NEMA) and the International Electrotechnical Commission (IEC).  The
two standards which most appropriately suit our needs are NEMA Enclosure Type
3R\autocite{nema3r}, and IEC International Protection Marking (IP) Code
25.\autocite{ip25}  These standards comprise a set of specifications which
should more than suitably cover our customer's initial requirement, ``It must be
able to survive in Knoxville, TN, weather."

%% @DocumentRequirement (6.1.4,"Describe the selected solution.")

Describe the solution.

\begin{description}
\item[From Wortham] DGPS and learning mode.
\item[From Chaloux] Ultrasonic sensors and obstacle avoidance.
\item[From Bouler] Solar power.
\item[From Bandara] Motors and batteries.
\end{description} 
%% @DocumentRequirement (6.1.5,"How did the team ensure that all requirements were met (to the extent that they were)?")
%% @DocumentRequirement (6.1.6,"How was the work at each phase (requirements, design, implementation, testing & evaluation) verified against the outcomes of prior phases?")
%% @DocumentRequirement (6.1.7,"Describe the results and/or outcomes of the project.")


6.1.5 How did the team ensure that all requirements were met (to the extent that they were)?

6.1.6 How was the work at each phase (requirements, design, implementation, testing \& evaluation) verified against the outcomes of prior phases?

6.1.7 Describe the results and/or outcomes of the project.

%% @DocumentRequirement (7,"Lessons Learned")
\section{Lessons Learned}

%% @DocumentRequirement (7.1,"Describe what went wrong or unexpected events and how the team adapted to these.")

We were fortunate enough to avoid any major mishaps throughout the project. In
forming our group it was important to all of us that our team remain small.
Delightfully our team of four has personalities that mesh well, and that has
enabled us to function as a cohesive unit with no internal strife.
%% @DocumentRequirement (7.2,"What would the team like to see done differently in future projects?")
If there is anything that could be improved upon we would have to choose
communication.  The four of us recognize that we each lead our own busy lives,
and as such we respect the fact that important emails or texts may occaisonally
go unanswered.


%% @DocumentRequirement (8,"Statement of Team Member's Contributions")
\section{Team Member Contributions}

\subsection{Administrative Roles}
%% @DocumentRequirement (8.1,"Describe the responsibilities of each team member.")
Our team lead is Doug Bouler. Doug is in charge of coordinating group
meetings, dividing up labor on assignments, maintaining contact with our
customer, and ensuring that deadlines are met.  Alex Wortham stepped up to the
plate to ensure that all of our group's written deliverables met the
requirements of the assignments.  His experience in \LaTeX\ was invaluable in
ensuring that our documents complied with the given style requirements.  The
decision to adopt \LaTeX\ for all of our project documentation was highly
conducive to the use of source control.  With that in mind Wortham set up and
maintained a Github project to host our documentation and allow for easy
collaboration as well as peace of mind that our documentation was safely backed
up in many locations.

\subsection{Other Roles}
These administrative roles aside, our team does not
believe in the strict assignment of roles to individuals.  When it comes to the
meat of all our work every member has contributed to all aspects of the project.
Through the use of this approach to our work we found the diversity in our
academic disciplines contributed to better outcomes than the work we were
capable of producing individually. 

%% @DocumentRequirement (8.2,"Describe the contributions of each team member.")

\clearpage
\subsection{Contributions}
The table below is a breakdown of each team member's contributions to each assignment.

\setstretch{1.1}
\begin{table}[h!]
\begin{center}
\begin{tabular}{|l|l|l|}
\hline
\textbf{Assignment} & \textbf{Role} & \textbf{Contributors} \\
\hline

Needs/Concept/Market Presentation & Writers & All \\
\cline{2-3}
& Editors  & All \\
\cline{2-3}
& Research  & Bandara \\
\hline

Needs \& Requirements Specification & Writers & All \\
\cline{2-3}
& Editors  & All \\
\cline{2-3}
& Research  & All \\
\hline

Analysis \& Solution Strategy Presentation & Writers & Bouler, Chaloux, Bandara \\
\cline{2-3}
& Editors  & Wortham \\
\cline{2-3}
& Research  & All \\
\hline

Analysis \& Solution Strategy Report & Writers & Bouler, Wortham \\
\cline{2-3}
& Editors  & Bandara, Chaloux \\
\cline{2-3}
& Research  & Bandara, Wortham \\
\hline

Final Poster & Writers & Wortham \\
\cline{2-3}
& Editors  & All \\
\cline{2-3}
& Design	& Wortham \\
\cline{2-3}
& Graphics	& Bouler \\
\hline

Final Paper & Writers & All \\
\cline{2-3}
& Editors  & All \\
\cline{2-3}
& Graphics	& All \\
\hline
\end{tabular}
\caption{Team Member Contributions}
\end{center}
\end{table}

\setstretch{2}
%% @DocumentRequirement (9,"Signatures of all Team Members and Customer.")
\clearpage
\section{Customer Agreement}

%% @DocumentRequirement (9.1,"Names, signatures, and dates.")

The undersigned agree that the requirements outlined in this document meet all needs for the product.

\signanddate{Vasile T. Garbulet}
\signanddate{Asanga Bandara}
\signanddate{Doug Bouler}
\signanddate{Alex Chaloux}
\signanddate{F. Alex Wortham}

%% @DocumentRequirement (9.2,"Dissenting statements (signed) -- if any.")

\end{document}
