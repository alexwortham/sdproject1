\documentclass[12pt,letterpaper]{article}
\usepackage[ampersand]{easylist}
\usepackage[margin=1in]{geometry}
\usepackage{requirements}
\usepackage{float}
\usepackage{amsmath}
\bibliography{Bibliography.bib}
%% @DocumentRequirement (1.1,"Title")
\title{Final Report}
%% @DocumentRequirement (1.3,"Team #")
\team{1}
%% @DocumentRequirement (1.4,"Team Members")
\author{Alex Chaloux, Alex Wortham, Asanga Bandara, Doug Bouler}
%% @DocumentRequirement (1.5,"Customer")
\customer{Vasile Tudor Garbulet}

\begin{document}

%% @DocumentRequirement (1,"Title Page")
%% @DocumentRequirement (1.2,"Date")
\reqstitlepage

%% @DocumentRequirement (2,"Table of Contents")
\tableofcontents
\clearpage
\setstretch{2}

%% @DocumentRequirement (3,"Executive Summary")
\section{Executive Summary}

%% @DocumentRequirement (3.1,"Concise statement of the need addressed and the project's goals.")
%% @DocumentRequirement (3.2,"Summary of the results and whether the goals were met.")
%% @DocumentRequirement (3.3,"Brief Discussion of challenges and outcomes.")

Our customer approached us with a need for an autonomous device capable of
safely and reliably cutting his grass. Our project goals were simple and
flexible. This device would need to be easily configurable and require little
maintenance. The device needs to be outside for several months out of the year
and therefore must be weatherproof. The mowing device must have the capability
to communicate wirelessly with a base component.  Another requirement was that
the device needed to have foreign object detection capabilities as well as at
least one method of obstacle avoidance. A major component of this project was
the insistence that this device needed to be electrically powered as well as
rechargeable.

Our team encountered many challenges in developing this device. First, we needed
a way to identify foreign objects in the path of the mower. After considering
various detection options, we decided that ultrasonic detection was the most
responsive for outdoor, energy-efficient detection. Next, we needed a way to
navigate the mower within inches of predetermined path. Unfortunately, standard
GPS is not reliable for precise measurements. Typically, these measurements can
only be accurate to within a three meter radius. After researching different
methods of localization, we found a method described as differential GPS. By
using two standard, inexpensive GPS units we would be able to achieve reliable
localization within a few centimeters. In order to weatherproof the base and
mower devices, we found international standards which would encompass the
protection of electronics against water damage. Cutting the grass was a
seemingly insignificant part of this project, but it was a component that we did
not take lightly. We researched several common blade types and decided upon a
method which uses small, pivoting blades within a rotating disc.

\clearpage
%% @DocumentRequirement (4,"Requirements")
\section{Requirements}

%% @DocumentRequirement (4.1,"Numbered list of requirements initially agreed to by team and customer.")
\begin{easylist}[articletoc] \requirements

& Operation Requirements

	&& The mower must be battery powered.

		&&& The battery must be rechargeable.

	&& The mower must be autonomous as outlined in requirement \ref{autonomous}.
	&& A home base will be provided for the mower.


& \label{user interaction}User Interaction Requirements

	&& The user may issue the following commands to the mower.

		&&& Begin mowing operations, as specified in requirement \ref{begin}.
		&&& Stop immediately, as specified in requirement \ref{stop}.
		&&& Return to base, as specified in requirement \ref{rtb}.

	&& The following state information will be provided to the user.

		&&& Current battery level.
		&&& Battery is charging.
		&&& Battery has completed charging.
		&&& Mowing operation is in progress.
		&&& Desired grass height.
		&&& Days of week the mower is scheduled to run.
		&&& Time of each day the mower is scheduled to run.
		&&& Mower operation has been forcefully stopped because of a violation in requirements outlined in \ref{environment} and \ref{safety}.

	
& \label{autonomous}Autonomous Functions Requirements

	&& The mower must return to base when it detects low battery level.
	&& \label{rtb}Return to base on command from interface.
	&& \label{stop}Stop immediately on command from interface. The following conditions define the stopped state.
		&&& Electrical power is disconnected from the wheels.
		&&& Electrical power is disconnected from the blades.
	&& \label{begin}Begin mowing on command from interface.
	&& Begin mowing on schedule.

& \label{environment}Operating Environment Requirements

	&& The product must be able to operate normally within an environment that

		&&& is above 32 degrees Fahrenheit.
		&&& is below 115 degrees Fahrenheit.
		&&& \label{incline limits}has ground which the mower detects is less than 33\% incline.
	&& The device will function as specified when it is not in contact with snow.
	&& The device will function as specified when it is not in contact with ice.

& \label{weather}Weather Resistance Requirements.
	
	&& The product's electrical components must be protected by a NEMA Type 3R enclosure. (\href{https://www.nema.org/Products/Documents/nema-enclosure-types.pdf}{https://www.nema.org/Products/Documents/nema-enclosure-types.pdf})
	&& The product's components must comply with International Protection Marketing Code 25 (IP25. \href{http://www.dsmt.com/pdf/resources/iprating.pdf}{http://www.dsmt.com/pdf/resources/iprating.pdf}).

& \label{safety}Safety Requirements

	&& The mower must stop immediately (requirement \ref{stop}) when...

		% These requirements need to be refined.
		&&& it senses an foreign object in its path. 
			&&&& A foreign object is any object small enough that the mower might be capable of running over it, such as garden hoses, small gardening tools, tennis balls, etc.
		&&& an emergency kill switch is pressed.
		&&& removed from its configured boundaries.

	&& The mower will sound an alarm when removed from its configured boundaries.

& Product Configuration Requirements

	&& All of the below must be configurable one or more times.

		&&& \label{boundaries}The user can set operating boundaries for the mower.
		&&& \label{desired height}The user can set desired height of grass.
		&&& The days of week the mower will run.

			&&&& Multiple days may be selected.

		&&& The time of day the mower will run.

			&&&& One time must be selected for each day.

& Budget Requirements

	&& The total cost of the project including, but not limited to, the following shall not exceed \$2,500.  % Here We can say,insted of having project cost, "The total cost of the product would be less than $2,500." ( something like that) 
		&&& Labor.
		&&& Prototypes.
		&&& Final product.
		&&& Unforeseen costs.

\end{easylist}

%% @DocumentRequirement (5,"Change Log")
\section{Change Log}

%% @DocumentRequirement (5.1,"Chronological list of any agreed-upon changes to the requirements.")
Our customer was satisfied with the initial set of requirements we developed, so
no changes were necessary.

%% @DocumentRequirement (6,"Documentation of the Design Process")
\section{Design Process}

% % @DocumentRequirement (6.1,"Narrative describing how the design proceeded
% from analysis of the requirements to final result.")
Upon analysis of the requirements, we decided that the requirements fell into
one of three categories:  user interaction, autonomous functions, and operating
environment.  Of these three categories, our primary concern was to make user
interaction with the product minimal and free of frustration.  Thus our
discussions began with the question, how do we want the user to interact with
our product?  This question shaped our thinking as we approached the rest of the
requirements.  The primary issue that came to light in researching our
competitors was product configuration.  Every product currently on the market
uses some form of boundary wires for configuration which must be installed in
very strict ways for proper operation of the product.  Next on the list was
obstacle detection and avoidance.  To our great surprise, no competitors on the
market have obstacle detection other than simple collision sensors.

The greatest impact of this lack of features is on the user.  The user must
spend a significant amount of time on product installation, and must clear the
area of all obstacles before each mowing cycle.  We quickly arrived at the
conclusion that this lack of features was due to a lack of intelligent
autonomous design in existing products.  Thus it became clear that in order to
relieve the user from these burdens we needed to focus our efforts on developing
 an intelligent autonomous design capable of tackling these issues.

We knew that if we could meet our objective with the autonomous design that the
remaining categories would pose comparatively less challenge in research and
development.  Nevertheless there were still some important issues that needed to
be addressed. All technological issues aside, we were lost when it came to the
issue of cutting grass.  We needed to know what type of blade would best suit
our design, knowing that this choice would have a major impact upon safety
concerns.  And finally, what standards can we implement for weather proofing?

%% @DocumentRequirement (6.1.1,"Describe how the effort was decomposed into manageable pieces to address the agreed-upon requirements.")
With these issues in mind we were able to develop a clear division of labor.  We
divided up our efforts into the categories of user experience, autonomous
design, and mechanical design.  Our lead software developer, Alex Wortham, in
cooperation with solutions architect, Alex Chaloux, took responsibility of
evaluating various autonomous solutions and their implications on user
experience.  Our team lead and hardware designer, Doug Bouler, looked into the
impacts that the mechanical design would have on our electronics platforms as
well as the implementation of these two.  These members' success was facilitated
by our skilled lead researcher and hardware tester, Asanga Bandara, who also
assisted all of us in our tasks.

%% @DocumentRequirement (6.1.2,"What open questions had to be answered by research?")
%% @DocumentRequirement (6.1.3,"What alternatives were explored?")
Of the many questions posed by these challenges, we were most perplexed by how
we would simplify product configuration.  Configuration of our competitor's
products takes in the order of hours, and it was our goal to reduce the
configuration process to the order of minutes.  The key to solving this complex
problem was solving the considerably more complex problem of high precision
localization of a mobile robotics platform.  Research efforts quickly revealed
there is no de facto standard for localization, and furthermore that there is no
simple solution to this highly complex problem.

Research turned up dozens of scholarly and IEEE papers detailing solutions with
varying levels of success but generally poor precision.  Precision localization
of our product is absolutely vital to compete with the precision of the boundary
wire solution.  Finally we found a solution which can be adapted to achieve the
necessary level of precision without adding high costs. \autocite{mobilegps}  We
believe that by supplementing the odometry found in this method with
accelerometers and gyroscopes we can achieve significantly improved drift
compensation. Granted this ability for high precision localization, we arrived
at a simple configuration process that relies on recording the robot's locations
along a path traversed in a user assisted learning mode.

Our next line of questioning was regarding obstacle avoidance.  What are the
best methods of obstacle detection?  Which methods can be implemented within our
cost constraints?  We considered options such as computer vision, infrared,
ultrasonic, and simple mechanical sensors.  Each of these options present their
own challenges, especially in outdoors environments.  Computer vision and
infrared both rely upon certain characteristic properties of light reflecting
off of obstacles in their detection range.  These properties are too easily
impacted by changes in lighting conditions which are inherent to outdoors
environments, making it difficult to implement a solution which can provide
consistent detection.

Our preliminary research suggests that ultrasonic sensors will be able to
provide the most consistent obstacle detection.  Simple mechanical sensors alone
have proven to be insufficient by our competitors. Even so our design will most
likely incorporate some form of mechanical sensors as a fallback detection
system, but more research on the shortcomings of ultrasonic sensors will be
required in order to assess what the best supplemental detection system would
be.

Another area of concern was the selection of a cutting blade which is effective
and safe.  Most existing autonomous products use a fixed blade system similar to
those found on most human operated mowers, and thereby inherit the same safety
concerns.  In our research we determined that this definitely is not the focus
area of innovation for autonomous mowers. The \textit{Husqvarna
Automower}\autocite{automower} however stands out with an innovative blade
design which, if allowed by applicable intellectual property laws, we would
borrow upon for our own design.

Weather proofing is a difficult subject with any product, as nothing can be
truly weather \textit{proof}.  In a search for a more reasonable, testable set
of requirements for weather \textit{resistance}, we sought out publicized
industry standards to suit our product's needs.  Research led us to two widely
accepted standards published by the National Electrical Manufacturers
Association (NEMA) and the International Electrotechnical Commission (IEC).  The
two standards which most appropriately suit our needs are NEMA Enclosure Type
3R\autocite{nema3r}, and IEC International Protection Marking (IP) Code
25.\autocite{ip25}  These standards comprise a set of specifications which
should more than suitably cover our customer's initial requirement, ``It must be
able to survive in Knoxville, TN, weather."

%% @DocumentRequirement (6.1.4,"Describe the selected solution.")
\subsection{Detailed Description of Individual Solutions}
Now to talk a little bit more in detail about how all of this is supposed to
come together we must first look more closely at our chosen solutions for power,
configuration, navigation, and drive components. 

\subsubsection{Power}
When considering the methods of recharging the battery of our mower, we decided
to first look at our competition's methods. Upon inspection, we realized that
all major autonomous mower-base designs required that the base be connected by
an electrical cord to an outdoor outlet. Our team decided that the presence of
sharp, rotating machinery coupled with a current-carrying wire was both an
electrical hazard as well as an enormous inconvenience to the user. After
considering other options, we decided that solar power, even in ambient light
would be sufficient for our purposes.\autocite{radiometry} Ambient light, even
in shadow or on cloudy days, has been shown to provide enough photons to a
photovoltaic cell to charge a battery given enough time to recharge. Because
this machine will only be operating every few days, the solar cell will have
more than enough time to recharge the battery. For example, a cloudy day
provides approximately 5000 lux of illuminance.
 
\begin{figure}[h!]
\[
	\dfrac{w}{m^2} \approx \dfrac{lux}{683}
\]
\caption{Illuminance Equation}
\end{figure}

Using the above equation, a square meter solar panel we can generate
around 8 Watts per hour under severe circumstances.\autocite{radiation} Both
batteries would be able to be fully charged within three days if the weather remained extremely cloudy. On a bright, sunny day,
which would provide up to 100,000 lux, both batteries could be charged within
just a few hours. We recommend that the customer places the base in a place that
has a high exposure to sunlight, but in the case that the mower is not receiving
enough power, we will provide a secondary mode of charging through a standard
electrical connection.

We decided that a battery with roughly 168Wh power capacity was sufficient for
the operation of our mowing device and that a battery with roughly 84Wh would be
satisfactory for our base battery. We needed the battery on the base component
to be able to power both the GPS as well as the interface and any communication
between the mower and the base. With our selected mower battery, we would have
at least three hours of operation given that the terrain was not strenuous to
the driving motors.

\subsubsection{Configuration}

Our chosen solution for configuration of the product guides the user through a
simple process.  Once learning mode has been enabled from the user
interface, the mower will guide the user through an iterative process.
\begin{enumerate}
  \item Guide the mower around the absolute boundaries of an area to mow.
  
  		The mower will be propelled under its own power, and the user will only need
  		to steer the mower by a push bar similar to an ordinary self propelled push
  		mower.  The only requirement for successful configuration is that the path
  		traversed by the user must be closed, meaning that the process will end when
  		the user arrives back at the position from which they began.  The user will
  		be notified by the mower when the path has been closed.
  		
  \item Guide the mower around any areas which are enclosed by the area to be
  mowed which should not be mowed.
  
  		The general use case for this is to guide the mower around large, permanent
  		obstacles inside the area to mow.  After completing the first step, users
  		will be prompted to outline these obstacles.  After placing the mower at the
  		edge of the obstacle, the user will push a button to begin path tracing. 
  		Once again, the mower must be guided in a closed path around the obstacle. 
  		The user will be notified when the path has been closed.
  
  \item (Optional) Add reachably disjoint areas to be mowed.
  
        A reachably disjoint area is an area which is not connected to any other area,
        but there exists a path by which the mower can reach it without
        assistance.  If a reachably disjoint area exists, the user
        may add the area by repeating steps 1 - 2, and then initiate a process to trace the path the mower
        will use to cross between disjoint areas.
        
  \item (Optional) Add unreachably disjoint areas to be mowed.
  
        Areas which are unreachably disjoint, meaning it is impossible for the
        mower to navigate between them without assistance, may be defined by
        repeating the above steps. The user will be notified during the mowing
        process when the mower needs assistance to reach these additional areas.
\end{enumerate}

\subsubsection{Navigation}

To facilitate this configuration process highly precise localization is needed.
Our answer to this problem is a custom solution fusing multiple existing
technologies to achieve accuracies within ten centimeters.  The basic idea for
our solution is derived from a differential GPS (DGPS) solution presented at
\textit{The Australian Conference on Robotics and Automation}.  This paper
entitled ``High Precision GPS Guidance of Mobile Robots'' details a unique
method of localization which ``\ldots\ gathers raw measurement data from two
[low cost] receivers and compares carrier phase data to obtain accuracies
normally associated with the much more costly Real Time Kinematics (RTK)
systems.''\autocite{mobilegps}  This however is not a complete solution to
localization because of the latencies associated with acquiring signal data
from low cost receivers.  

Some form of fallback guidance must take control
during the acquisition periods and any periods in which a location cannot be
acquired.  The solution presented in this paper uses odometry, measuring the
current speed of travel, to interpolate intermediary location data.  The problem
with this is that any slippage of the wheels introduces significant error
which cannot be corrected until the next location is acquired from GPS.
Our proposed solution to this is to use a combination of accelerometers and
gyroscopes for interpolating location data.  These sensors can be acquired for
very low cost, and even in a package which combines readings from the two into
acceleration vectors that can be read by a microcontroller such as an Arduino. 
These readings can be read at X frequency, which will present almost
instantaneous changes in the acceleration vectors that can be used to greatly
reduce error introduced by the slippage problem.
  
During the configuration process location data acquired from our custom
localization solution will be recorded to internal storage in learning mode, and
will be used during the mowing process to help the mower navigate through its
configured area.  It is widely known that in machine learning systems even the
smallest change in the environment can render the machine incapable of
completing the task it was trained to do.  In order to combat this problem, we
need to incorporate a sophisticated obstacle avoidance system to help the mower
navigate around obstacles which were not defined during the configuration
process.
\subsubsection{Obstacle Avoidance}

During a mowing session, there is a high probability that there will be
obstacles in the way of the mower that were not defined in the configuration
process. These could include items left in the yard, such as toys or balls, or
moving things, such as pets or children. We broke down our obstacles into two
categories: obstacles that will be above the surface of the grass, and obstacles
that will be below the surface of the grass. Obstacles above the surface of the
grass are easily detected, and are therefore easily avoidable. In order to
successfully mow the yard, these obstacles need to be avoided and the mower
needs to be able to move on to unblocked areas of the yard. However, small
obstacles below the surface of the grass can end up underneath our mower, so we
need to minimize the damage that our mower will inflict on these obstacles.

To accomplish this, we decided that ultrasonic sensors would be best suited to
discovering obstacles, moving or stationary, above the surface of the grass.
When our mower encounters an obstacle, it will stop and wait for the obstacle to
move. If the obstacle is stationary, the mower will maneuver around the
obstacle, and continue mowing areas of the yard that are unblocked. Our mower
will continue attempting to mow a blocked area each time it is encountered, so
that if the area is now unblocked it can be successfully mowed. If any area is
blocked for the entire mowing session, the mower will notify the user so that
the obstacle can be moved before the next time the lawn is mowed. In this
manner, our mower will avoid objects that can be detected above the surface of
the grass.

However, there are also going to be obstacles that are below the surface of the
grass, and therefore cannot be picked up by ultrasonic sensors. These objects
would include small toys, such as a baseball, or items such as a rake. These are
obstacles that our mower will be able to move over, and could therefore come
into contact with the blades. We realized that we will be unable to stop our
mower from hitting every shape and size of obstacle, due to the nature of the
device, so we needed to find a way to minimize the damage that would be dealt to
the mower and the encountered objects.

To accommodate this, we found a blade design that included a rotating disk, and
small lightweight blades. The blades would be pinned onto the disk, such that
they could rotate around their pin independently of the disk rotation. When the
disk is rotating at a high speed, the rotational inertia will cause the blades
to rotate out, with the sharp edge facing perpendicular to the disk. If an
object with substantial resistance is encountered, the blade will rotate
backward on contact, and will not be forced through an object like a traditional
mower blade. Therefore, the blades and encountered obstacles will both sustain
less total damage, allowing for a safer mowing device.

\subsubsection{Drive Components}

We have chosen to use two independently driven electric motors to power the rear
wheels.  This will give the mower zero-turn steering capabilities which will
allow it to reach areas with tight corners.  In order to select the right motors
for this application we needed to consider specifications for torque, power,
speed, and weight.  After evaluating the types of motors that are available, we
decided that direct current (DC) brushless gearmotors with planetary gearing
were best suited for this application.  Planetary gear motors have several
advantages over their counterparts.
\begin{singlespace}
\begin{itemize}
  \item Compact size and weight, allowing them to be up to 50\% smaller with the
  same torque output.
  \item High power density.
  \item Longer gear life than their counterparts under similar loads.
  \item High efficiency.
  \item Coaxial arrangement, meaning no offset output shaft.
  \item Modularity, allowing them to be stacked.
\end{itemize}
\end{singlespace}

%% @DocumentRequirement (6.1.5,"How did the team ensure that all requirements were met (to the extent that they were)?")

% Not applicable

%% @DocumentRequirement (6.1.6,"How was the work at each phase (requirements, design, implementation, testing & evaluation) verified against the outcomes of prior phases?")

% Not applicable

%% @DocumentRequirement (6.1.7,"Describe the results and/or outcomes of the project.")

\subsection{Results}


Building a safe, cost-effective, easy to use autonomous mower poses a
significant design challenge.  We as a team have put together an outline for an
ideal product that rises to the challenge, and sought to incorporate these
ideals into our design process.  Though it was initially a daunting task,
diligent research and cohesive teamwork brought to fruition a practical design
for an autonomous mower.  We sought to keep the design as simple as possible
while still meeting our requirements to maintain a realistic design based on
proven technologies wherever possible.  We seek to innovate primarily in the
area of user experience by developing a more intelligent autonomous mower that
is easily configurable, requires minimal maintenance, and operates safely in a
somewhat dynamic environment without the need for supervision.


%% @DocumentRequirement (7,"Lessons Learned")
\section{Lessons Learned}

%% @DocumentRequirement (7.1,"Describe what went wrong or unexpected events and how the team adapted to these.")

We were fortunate enough to avoid any major mishaps throughout the project. In
forming our group it was important to all of us that our team remain small.
Delightfully our team of four has personalities that mesh well, and that has
enabled us to function as a cohesive unit with no internal strife.
%% @DocumentRequirement (7.2,"What would the team like to see done differently in future projects?")
If there is anything that could be improved upon we would have to choose
communication.  The four of us recognize that we each lead our own busy lives,
and as such we respect the fact that important emails or texts may occasionally
go unanswered.


%% @DocumentRequirement (8,"Statement of Team Member's Contributions")
\section{Team Member Contributions}

\subsection{Administrative Roles}
%% @DocumentRequirement (8.1,"Describe the responsibilities of each team member.")
Our team lead is Doug Bouler. Doug is in charge of coordinating group
meetings, dividing up labor on assignments, maintaining contact with our
customer, and ensuring that deadlines are met.  Alex Wortham stepped up to the
plate to ensure that all of our group's written deliverables met the
requirements of the assignments.  His experience in \LaTeX\ was invaluable in
ensuring that our documents complied with the given style requirements.  The
decision to adopt \LaTeX\ for all of our project documentation was highly
conducive to the use of source control.  With that in mind Wortham set up and
maintained a Github project to host our documentation and allow for easy
collaboration as well as peace of mind that our documentation was safely backed
up in many locations.

\subsection{Other Roles}
These administrative roles aside, our team does not
believe in the strict assignment of roles to individuals.  When it comes to the
meat of all our work every member has contributed to all aspects of the project.
Through the use of this approach to our work we found the diversity in our
academic disciplines contributed to better outcomes than the work we were
capable of producing individually. 

%% @DocumentRequirement (8.2,"Describe the contributions of each team member.")

\clearpage
\subsection{Contributions}
The table below is a breakdown of each team member's contributions to each assignment.

\setstretch{1.1}
\begin{table}[h!]
\begin{center}
\begin{tabular}{|l|l|l|}
\hline
\textbf{Assignment} & \textbf{Role} & \textbf{Contributors} \\
\hline

Needs/Concept/Market Presentation & Writers & All \\
\cline{2-3}
& Editors  & All \\
\cline{2-3}
& Research  & Bandara \\
\hline

Needs \& Requirements Specification & Writers & All \\
\cline{2-3}
& Editors  & All \\
\cline{2-3}
& Research  & All \\
\hline

Analysis \& Solution Strategy Presentation & Writers & Bouler, Chaloux, Bandara \\
\cline{2-3}
& Editors  & Wortham \\
\cline{2-3}
& Research  & All \\
\hline

Analysis \& Solution Strategy Report & Writers & Bouler, Wortham \\
\cline{2-3}
& Editors  & Bandara, Chaloux \\
\cline{2-3}
& Research  & Bandara, Wortham \\
\hline

Final Poster & Writers & Wortham \\
\cline{2-3}
& Editors  & All \\
\cline{2-3}
& Design	& Wortham \\
\cline{2-3}
& Graphics	& Bouler \\
\hline

Final Paper & Writers & All \\
\cline{2-3}
& Editors  & All \\
\cline{2-3}
& Graphics	& All \\
\hline
\end{tabular}
\caption{Team Member Contributions}
\end{center}
\end{table}

\setstretch{2}
%% @DocumentRequirement (9,"Signatures of all Team Members and Customer.")
\clearpage
\section{Customer Agreement}

%% @DocumentRequirement (9.1,"Names, signatures, and dates.")

The undersigned agree that the requirements outlined in this document meet all needs for the product.

\signanddate{Vasile T. Garbulet}
\signanddate{Asanga Bandara}
\signanddate{Doug Bouler}
\signanddate{Alex Chaloux}
\signanddate{F. Alex Wortham}

%% @DocumentRequirement (9.2,"Dissenting statements (signed) -- if any.")

\end{document}
