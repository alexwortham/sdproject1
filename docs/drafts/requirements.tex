% Requirements Document for Project 1
%
% CAUTION!!! Only edit stuff between ``EDIT BELOW THIS" and 
% ``EDIT ABOVE THIS" unless you know what you are doing!
%
% Use & to set the level of the requirement. White space does not matter.
% Lines which begin with % are comments and are not shown in the output.
%
% EXAMPLE:
% & Super Awesome Requirements
% && The product must be totally awesome
% &&& So awesome that it will blow your mind
%
% Will produce something like...
% 1  Super Awesome Requirements
%    1.1  The product must be totally awesome
%         1.1.1  So awesome that it will blow your mind
%
% When you need to reference one requirement from another, first label
% the target requirement with \label{aName} and then you can get that
% requirement's number with \ref{aName}. label need not occur before ref,
% and you must compile the document TWICE to see updated references.
%
% EXAMPLE:
% & \label{awesome} Super Awesome Requirements
% && The product must be totally awesome
% &&& So awesome that it will blow your mind 
% & Super Cool Requirements
% && In addition to the awesomeness described in requirement \ref{awesome}, the following coolness is required
% &&& The product must be super cool
%
% Will produce something like...
% 1  Super Awesome Requirements
%    1.1  The product must be totally awesome
%         1.1.1  So awesome that it will blow your mind 
% 2  Super Cool Requirements
%    2.1  In addition to the awesomeness described in requirement 1, the following coolness is required
%         2.1.1  The product must be super cool
%
% GOTCHAS:
%
% $, #, &, % are special characters in LaTeX, you must escape them with \
% Use \$, \#, \&, \% respectively.
% 
% Do not use ^ or _, or if you must consult the almighty Google for help.
% 
% Quotes are weird in LaTeX, ' and " produce right quotes, 
% ` and `` produce left quotes. E.G. `single quotes' ``double quotes".

\documentclass[12pt,letterpaper]{article}
\usepackage[ampersand]{easylist}
\usepackage[margin=1in]{geometry}
\newcommand\requirements{\ListProperties(Space=1ex,Space*=.5ex)}
\author{Alex Wortham, Alex Chaloux, Doug Bouler, Asanga Bandara}
\title{Requirements}
\begin{document}

\begin{center}
{\LARGE Requirements}
\end{center}

\begin{easylist}[articletoc] \requirements

%%%%%%%%%%%%%%%%%%%%%%%%%
%    EDIT BELOW THIS    %
%%%%%%%%%%%%%%%%%%%%%%%%%

& Project Requirements

&& The total cost of the below shall not exceed \$2,500. //Initially I thought that ``total project cost" was too vague, but maybe this is too specific?
&&& Labor.
&&& Prototypes.
&&& Final product.
&&& Unforeseen costs.

& Operation Requirements

&& The mower must be battery powered.
&&& The battery must be rechargeable.
&& The mower must be autonomous as outlined in \ref{autonomous}
&& A home base will be provided for the mower, which will serve as...
&&& the starting point for mowing operations.
&&& the ending point for mowing operations.
&&& the charger for the mower.
&& Operational safety requirements are outlined in \ref{safety}

& \label{autonomous} Autonomous Functions Requirements

&& The mower must return to base when battery level is low.
&&& The mower must connect itself to the base for recharging.
&& Return to base on command. See x.x

& Operating Environment Requirements

&& \label{environment} The product's components must be able to operate normally within an environment that
&&& is above XX degrees Fahrenheit 
&&& is below XX degrees Fahrenheit
&& The product's components must not become inoperable due to constant exposure to conditions within the limits specified in \ref{environment}, as well as
&&& Rainfall not to exceed XX in/hr.
&&& Rain accumulation not to exceed XX inches.
&& The mower will only cut grass on level ground not to exceed 25 degrees in incline.

& \label{safety} Safety Requirements

&& The mower must power down when...
&&& it senses an ``undefined obstacle" (this is too vague) in its path. 
&&& it senses moving objects entering its ``safety bubble."
&&& an emergency kill switch on mower unit is pressed.
&&& an emergency kill switch on the base / control panel is pressed.
&& The mower will sound an alarm when removed from its configured boundaries.
&& The mower will cease to function when removed from its configured boundaries.

& Product Configuration Requirements

&& All of the below may be configured one or more times.
&& Operating boundaries
&&& The user can set boundaries for the mower.
&&& Boundary is defined as a line the mower will not cut grass across.
&&& The user must configure a set of boundaries which comprise a closed shape in order for the mower to operate.
&& Depth of cut
&&& Defined as the the desired height of the grass after it has been cut.
&&& Adjustable to a minimum of XX inches.
&&& Adjustable to a maximum of XX inches.
&&& Is accurate to plus or minus 0.XX inches.

& And the Rest //Needs editing below here

&& Separate power/control systems for propulsion and blades //this is implementation

&& Mower must be started manually by physical switch

&& state display

%%%%%%%%%%%%%%%%%%%%%%%%%
%    EDIT ABOVE THIS    %
%%%%%%%%%%%%%%%%%%%%%%%%%

\end{easylist}


\end{document}
