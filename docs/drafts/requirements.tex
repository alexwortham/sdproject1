% Requirements Document for Project 1
%
% CAUTION!!! Only edit stuff between ``EDIT BELOW THIS" and
% ``EDIT ABOVE THIS" unless you know what you are doing!
%
% Use & to set the level of the requirement. White space does not matter.
% Lines which begin with % are comments and are not shown in the output.
%
% EXAMPLE:
% & Super Awesome Requirements
% && The product must be totally awesome
% &&& So awesome that it will blow your mind
%
% Will produce something like...
% 1  Super Awesome Requirements
%    1.1  The product must be totally awesome
%         1.1.1  So awesome that it will blow your mind
%
% When you need to reference one requirement from another, first label
% the target requirement with \label{aName} and then you can get that
% requirement's number with \ref{aName}. label need not occur before ref,
% and you must compile the document TWICE to see updated references.
%
% EXAMPLE:
% & \label{awesome} Super Awesome Requirements
% && The product must be totally awesome
% &&& So awesome that it will blow your mind
% & Super Cool Requirements
% && In addition to the awesomeness described in requirement \ref{awesome}, the following coolness is required
% &&& The product must be super cool
%
% Will produce something like...
% 1  Super Awesome Requirements
%    1.1  The product must be totally awesome
%         1.1.1  So awesome that it will blow your mind
% 2  Super Cool Requirements
%    2.1  In addition to the awesomeness described in requirement 1, the following coolness is required
%         2.1.1  The product must be super cool
%
% GOTCHAS:
%
% $, #, &, % are special characters in LaTeX, you must escape them with \
% Use \$, \#, \&, \% respectively.
%
% Do not use ^ or _, or if you must consult the almighty Google for help.
%
% Quotes are weird in LaTeX, ' and " produce right quotes,
% ` and `` produce left quotes. E.G. `single quotes' ``double quotes".

\documentclass[12pt,letterpaper]{article}
\usepackage[ampersand]{easylist}
\usepackage[margin=1in]{geometry}
\newcommand\requirements{\ListProperties(Space=1ex,Space*=.5ex)}
\author{Alex Wortham, Alex Chaloux, Doug Bouler, Asanga Bandara}
\title{Requirements}
\begin{document}

\begin{center}
{\LARGE Requirements}
\end{center}

\begin{easylist}[articletoc] \requirements

%%%%%%%%%%%%%%%%%%%%%%%%%
%    EDIT BELOW THIS    %
%%%%%%%%%%%%%%%%%%%%%%%%%

& Project Requirements

&& The total cost of the project including, but not limited to, the following shall not exceed \$2,500. %( I suggest to put this idea at the very end of the document )
&&& Labor.
&&& Prototypes.
&&& Final product.
&&& Unforeseen costs.

& Operation Requirements

&& The mower must be battery powered.
&&& The battery must be rechargeable.
&& The mower must be autonomous as outlined in requirement \ref{autonomous}.
&& The mower must be started manually.
&& A home base will be provided for the mower, which will serve as...%( initial location for starting, ending and battery charging.) ( I suggest to remove the following bulleting. ) 
&&& the starting point for mowing operations.
&&& the ending point for mowing operations.
&&& the charger for the mower's batteries.
&& Operational safety requirements are outlined in requirement \ref{safety}.%// I think this is repeating. As Dr. Birdwell said, we don�t want to repeat ( Lecture 04- slide 10) we are writing about the safety under another requirement, therefore, we don�t want to mention it here. 
& \label{user interaction} User Interaction Requirements

&& The user may issue the following commands to the mower.
&&& Begin mowing operations, as specified in requirement \ref{begin}.
&&& Map terrain, as specified in requirement \ref{map}.
&&& Stop immediately, as specified in requirement \ref{stop}.
&&& Return to base, as specified in requirement \ref{rtb}.

&& The base will provide the following state information to the user. % do we need to specify like this for this document ? 
&& Current battery level in percentage.
&& Battery is charging.
&& Battery has completed charging.
&& Battery level is low.
&&& Battery level is considered to be low when the mower cannot complete its operation before power is exhausted.
&& Mowing operation is in progress.
&& Mower operation has been forcefully stopped because of a violation in  requirements outlined in requirements \ref{environment} and \ref{safety}.
&&& If the mower has been stopped, the cause of the stop.

& \label{autonomous} Autonomous Functions Requirements

&& The mower must return to base when battery level is low.
&& \label{rtb} Return to base.
&&& The mower must immediately disengage cutting device.
&&& The mower must navigate to base.
&&& The mower must connect itself to the charger. % I think this is too specific, if we can remove this part from the requirement for right now. 
&& \label{stop} Stop immediately.
&&& The mower must immediately disengage cutting device.
&&& The mower must brake to a complete stop.
&&& The mower must cut all power.
&& \label{map} Map terrain.		
&&& ... \\ % double backslash causes an immediate line break
\vdots %remove this line when done
&& \label{begin} Begin mowing.	% i think this part don't need
&&& ... \\% double backslash causes an immediate line break
\vdots %remove this line when done

& \label{environment} Operating Environment Requirements

&& The product's components must be able to operate normally within an environment that
&&& is above XX degrees Fahrenheit
&&& is below XX degrees Fahrenheit
&& The product's components must not become inoperable due to constant exposure to conditions within the limits specified in requirement \ref{environment}.
&& The mower will comply with the following standards for ``weather proofing." % I think this data is not necessary for this documents. this is not a customer's specific requirement we can talk about this in later documents.
&&& Standard XXX. 
&&& Standard YYY.
&&& Standard ZZZ.
&& \label{incline limits} The mower will only traverse ground which it detects is less than 33\% incline as measured across its wheel base (from center to center of the wheels).
&& \label{spatial limits} The mower will only traverse spaces which it detects to be greater than
&&& XX inches in length.
&&& XX inches in width.
&&& XX inches in height.

& \label{safety} Safety Requirements

&& The mower must stop immediately (requirement \ref{stop}) when...
% this is too vague.
&&& it senses an ``undefined obstacle" in its path.
&&& it senses moving objects entering its ``safety bubble."
&&& an emergency kill switch on mower unit is pressed.
&&& an emergency kill switch on the base / control panel is pressed. %( I think these two have the same idea. I suggest only one about emergency operation ) 
&& The mower will sound an alarm when removed from its configured boundaries.
&& The mower will cease to function when removed from its configured boundaries.

& Product Configuration Requirements

&& All of the below must be configurable one or more times.
&& \label{boundaries} Operating boundaries
&&& The user can set boundaries for the mower.
&&& Boundary is defined as a line the mower will not cut grass across.
&&& The user must configure a set of boundaries which comprise a closed shape in order for the mower to operate.
&& \label{desired height} Desired height of grass after cut.
&&& Must be measurable by the mower.
&&& Must be adjustable to a minimum of XX inches.
&&& Must be adjustable to a maximum of XX inches.

& Mowing Requirements

&& A mowing operation will be complete when the following conditions are met.
% I think it's really difficult to put an exact percentage on this... Sure it's testable but guaranteeing a firm percentage might be dangerous.
&&& At least 98\% of the configured area (requirement \ref{boundaries}) has been cut, not including areas subject to limitations outlined in requirements \ref{incline limits} and \ref{spatial limits}.
&&& The height of the grass which has been cut is accurate to plus or minus 0.XX inches of the desired height (requirement \ref{desired height}).



%%%%%%%%%%%%%%%%%%%%%%%%%
%    EDIT ABOVE THIS    %
%%%%%%%%%%%%%%%%%%%%%%%%%

\end{easylist}


\end{document}
